Modelar ambientes virtuais é uma tarefa árdua, podendo requerer grande tempo e esforço da parte dos artistas, se estes optarem por gerar cada objeto manualmente. Baseado nisto, a modelagem procedural surgiu com a proposta de trazer alguns benefícios no que se refere à geração das diversas camadas de ambientes virtuais, como vegetação, terrenos, estradas, rios, edifícios e cidades, por exemplo. Tais modelos, por sua vez, podem ser aplicados em diversos cenários, como planejamento urbano, jogos, filmes, simulações, entre outros. Contudo, também surgiram alguns desafios, como a falta de intuitividade na utilização de alguns \textit{frameworks} existentes, a semântica em relação à disposição dos elementos nos modelos, o grau de realismo com que eles são apresentados, e a geração de formas mais complexas, como estruturas arredondadas. Diversas pesquisas tentam mitigar tais dificuldades, assim, seguindo esta premissa, o presente trabalho descreve uma abordagem pioneira para a resolução do problema da geração de modelos arquiteturais com geometria arredondada utilizando \gls{SELEX}, por meio da aplicação de técnicas de deformação, permitindo o arredondamento de estruturas no sentido externo e interno.

% Separe as palavras-chave por ponto
\palavraschave{Ambientes virtuais. Modelagem procedural. Modelagem arquitetural. Gramáticas. \textit{Selection Expressions}. Deformação.}