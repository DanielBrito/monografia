Modeling virtual environments is an arduous task, and might require great time and effort on the part of the artists if they choose to generate each object manually. Based on this, the procedural modeling came up with the proposal to bring some benefits with regard to the generation of the various layers of virtual environments, such as vegetation, land, roads, rivers, buildings, and cities, for example. Such models, in turn, can be applied in various scenarios, such as urban planning, games, movies, simulations, among others. However, some challenges also emerged, such as the lack of intuitiveness in the use of some existing frameworks, the semantics in relation to the layout of elements in the models, the degree of realism with which they are displayed, and the generation of more complex forms, such as rounded structures. Several researches try to mitigate these challenges, thereby following this premise, the present work describes a pioneering approach to solve the problem of generating architectural models with rounded geometry using \gls{SELEX}, through the application of deformation techniques, allowing the rounding of structures in the external and internal directions.

% Separe as Keywords por ponto
\keywords{Virtual environments. Procedural modeling. Architectural modeling. Grammars. Selection Expressions. Deformation.}