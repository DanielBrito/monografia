\chapter{Introdução}
\label{cap:introducao}

Neste capítulo, serão apresentados alguns contextos nos quais a técnica de modelagem procedural é aplicada, bem como as motivações e os objetivos do presente trabalho.

\section{Contextualização}
\label{sec:contextualização}

Nos últimos anos, a modelagem procedural tem sido um eminente tópico de pesquisa, devido ao fato de que simulações de realidade virtual, jogos e filmes, têm se tornado cada vez mais prevalecentes. Na indústria cinematográfica, boa parte das produções utiliza recursos de Computação Gráfica. Um exemplo é o filme \textit{Avatar}, produzido e dirigido por James Cameron, sendo a primeira obra a contar com um mundo 3D foto-realista totalmente gerado por computador: o planeta Pandora \cite{simon2011}. A indústria de jogos, por sua vez, ultrapassou a indústria do cinema, e alguns \textit{video games} têm orçamentos maiores do que os sucessos de bilheteria de Hollywood \cite{teboul2011}. Um dos exemplos que mais se destaca é a série \textit{Grand Theft Auto}, que combina comportamentos quase totalmente irrestritos em ambientes virtuais inspirados em cidades dos Estados Unidos, como Nova Iorque e Miami \cite{simon2011}.

A modelagem procedural é aplicada em uma grande variedade de áreas, como na geração de texturas, plantas, terrenos, edifícios, cidades, estradas, malhas fluviais, entre outras. A geração de edifícios, em particular, é uma das mais desenvolvidas, possuindo métodos que podem ser amplamente empregados na criação de modelos detalhados e realistas \cite{smelik2014}. 

Além dos segmentos da indústria citados anteriormente, a modelagem procedural também pode ser utilizada em planejamento urbano, análises logísticas e simulações, uma vez que representações realistas de um espaço urbano podem ser aplicadas em treinamentos de socorro, análise de rotas, planos de evacuação, mapeamento de tráfego, entre outras situações \cite{francisco2014}.

Seja qual for o cenário, a atividade de criar um ambiente virtual é bastante complexa, visto que existe uma crescente demanda pela modelagem de arquiteturas com características cada vez mais próximas da realidade. Portanto, técnicas de modelagem procedural estão em constante evolução, buscando atender às necessidades emergentes de artistas e animadores.

\section{Justificativa}
\label{sec:justificativa}

A geração procedural de edifícios pode reduzir de maneira significativa os custos de modelagem, uma vez que permite a produção de uma variedade de formas semelhantes a partir de um conjunto de regras generativas \cite{smelik2014}, sendo amplamente aplicada na criação de fachadas de prédios retos. Entretanto, para geração de estruturas arquitetônicas mais complexas, com geometria arredondada, por exemplo, algumas técnicas trabalham apenas por meio da sua importação, como complemento.

Baseado nisto, o presente trabalho tem como principal contribuição apresentar uma abordagem, baseada em técnicas de deformação, cujo intuito é resolver a limitação da geração de modelos arquiteturais com geometria arredondada, por meio da utilização de \textit{Selection Expressions}, uma linguagem de modelagem procedural relativamente recente, que representa uma evolução do conceito de \textit{\gls{CGA}}, utilizado nas clássicas \textit{CGA Shape} e \textit{CGA++}.

\section{Objetivos}
\label{sec:objetivos}

\subsection{Objetivo geral}
\label{sec:objetivos_gerais}

Gerar modelos arquiteturais com geometria arredondada utilizando \textit{Selection Expressions}, por meio da especificação de uma nova operação de deformação.

\subsection{Objetivos específicos}
\label{sec:objetivos_especificos}

\begin{itemize}
    \item Implementar e avaliar a linguagem para geração de modelos de massa com arquitetura arredondada, ou seja, com bordas suavizadas;
    \item Integrar e avaliar linguagem com ferramenta de modelagem 3D por meio de \textit{scripts};
    \item Avaliar a aplicação de técnicas de deformação para criação de modelos arquiteturais com geometria arredondada;
    \item Avaliar o resultado obtido frente a alguns exemplos do mundo real.
\end{itemize}

\section{Estrutura do trabalho}
\label{sec:estrutura_trabalho}

O presente trabalho está disposto em 6 capítulos. No Capítulo \ref{cap:fundamentacao-teorica}, por meio de uma fundamentação teórica, são apresentados conceitos e técnicas do estado da arte, para o melhor entendimento dos capítulos seguintes. No Capítulo \ref{cap:trabalhos-correlatos}, são introduzidas ideias pertinentes à resolução do problema a ser abordado, com base em trabalhos correlatos. No Capítulo \ref{chap:proposta}, descreve-se o problema e uma possível abordagem para resolvê-lo. No Capítulo \ref{chap:resultados}, são apresentados alguns modelos obtidos com a técnica proposta, bem como uma breve análise estatística dos resultados. Por fim, no Capítulo \ref{chap:conclusoes}, são apresentadas as últimas considerações e possíveis tópicos para trabalhos futuros.